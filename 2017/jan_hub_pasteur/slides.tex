\documentclass{beamer}

\usepackage[T1]{fontenc}
\usepackage[utf8]{inputenc}
\usepackage{lmodern}
\usepackage{graphicx}
\usepackage[absolute,overlay]{textpos}
\usepackage{multicol}
\usepackage{listings}
\usepackage{svg}

%% Beamer customization--------------------------------------------------------

\usepackage{xcolor}




\usetheme{Warsaw}

%% Themes
% Outer themes
\useoutertheme{shadow}
% Rounded boxes and shadows
\useinnertheme[shadow=true]{rounded}
% Solid \item symbols
\useinnertheme{circles}

%% Custom colors
\definecolor{rltgreen}{rgb}{0,0.5,0}
\definecolor{pasteur}{RGB}{0,90,154}
\setbeamerfont{block title}{size={}}
\setbeamercolor{structure}{fg=pasteur}
\setbeamercolor{item}{fg=pasteur}

%Color of title
\setbeamertemplate{frametitle}
{
    \nointerlineskip
    \begin{beamercolorbox}[sep=0.3cm,ht=1.8em,wd=\paperwidth]{frametitle}
        \vbox{}\vskip-2ex%
        \strut\insertframetitle\strut
        \vskip-0.8ex%
    \end{beamercolorbox}
}
% Hide navigation symbols
\setbeamertemplate{navigation symbols}{}

%% Title block
\setbeamercolor*{title}{use=structure,fg=white,bg=pasteur}

\makeatletter

%% Top infolines
\setbeamertemplate{headline}{%
\leavevmode%
  \hbox{%
    \begin{beamercolorbox}[wd=\paperwidth,ht=2.5ex,dp=1.125ex]{palette quaternary}%
    \insertsectionnavigationhorizontal{\paperwidth}{}{\hskip0pt plus1filll}
    \end{beamercolorbox}%
  }
}


%% Define Snakemake -----------------------------------------------------------

\definecolor{eclipseBlue}{RGB}{42,0.0,255}
\definecolor{eclipseGreen}{RGB}{63,127,95}
\definecolor{eclipsePurple}{RGB}{127,0,85}

\lstset{language=Python}
\lstset{
    basicstyle=\scriptsize\ttfamily,
    morekeywords={rule, output, shell, params, run, configfile, temp, log},
    showstringspaces=false,
    commentstyle=\color{eclipseGreen}, % style of comments
    keywordstyle=\color{eclipsePurple}, % style of keywords
    stringstyle=\color{eclipseBlue}, % style of strings
}



%% Set up title ---------------------------------------------------------------

\title[Sequana]{A very brief history of Sequana}
\author[T. Cokelaer]{Thomas Cokelaer}
\institute{Institut Pasteur}
\date{Jan 17th 2017, Institut Pasteur (HUB talk)}


\AtBeginSection[]{
  \begin{frame}
  \vfill
  \centering
  \begin{beamercolorbox}[sep=8pt,center,shadow=true,rounded=true]{title}
    \usebeamerfont{title}\insertsectionhead\par%
  \end{beamercolorbox}
  \vfill
  \end{frame}
}

\begin{document}

%% Title slide ----------------------------------------------------------------

\begin{frame}[plain]
    \titlepage
    \begin{textblock*}{5cm}(4.5cm,0.3cm)
        \includegraphics[scale=0.09]{images/Institut_Pasteur.png}
    \end{textblock*}
\end{frame}

%% Slides ---------------------------------------------------------------------

\section{Motivation}
\begin{frame}
    \frametitle{NGS at Biomics (Sean Kennedy)}

 Development driven by the Biomics Pole at Pasteur Institute, which involves
 many aspects of NGS including :

 \tiny
 \begin{block}{https://research.pasteur.fr/en/team/biomics/}
  \begin{itemize}
  \item De novo and targeted sequencing of viruses, prokaryotes and eukaryotes
  \item Variant (SNP, indel, large rearrangements) detection
  \item Transcriptional analysis (RNA-Seq) for both prokaryotes and eukaryotes
  \item 16S and deep-sequencing metagenomic studies (mouse, human, and other
environments)
  \item Epigenetics (CHIP-Seq, methylation studies)
  \end{itemize}
 \end{block}
 \small
\end{frame}

\begin{frame}
 \frametitle{Needs}

    \begin{block}{What do we have \dots or not ?}
     \includegraphics[scale=0.05]{../../images/positive.png}\; A bunch of pipelines dedicated
to NGS data\\
     \includegraphics[scale=0.05]{../../images/positive.png}\; Expertise\\
     \includegraphics[scale=0.05]{../../images/negative.png}\; Lack of \\
    \begin{itemize}
     \item traceability ?
     \item reproducibility ?
     \item co-development ?
     \item common framework ?
    \end{itemize}
    \end{block}

    \begin{block}{What do we need ?}
    \begin{itemize}
     \item A framework to combine or re-use existing pipelines
     \item Fast development (iterative process)
     \item Continuous Integration and Quality Software (reproducibility, 
           traceability, test, documentation)
    \end{itemize}
    \end{block}
\end{frame}


\begin{frame}
    \frametitle{Why Sequana ?}
    
    \begin{block}{Enforce a common framework}
    \begin{itemize}
        \item Using Snakemake as a common language to design new pipelines
        \item Provide reusable snakemake rules and modules
    \end{itemize} 
     \end{block}

    \begin{block}{A toolbox in sequana to parse and analyse various data sets }
    \begin{itemize}
        \item Include pandas for data mining
        \item matplotlib for further visualisation
    \end{itemize} 
    \end{block}

    \begin{block}{A set of reports to improve}
    \begin{itemize}
        \item Software Quality
        \item Diffusion
        \item reproducibility
    \end{itemize} 
    \end{block}
    
\end{frame}





\section{Snakemake as a workflow manager}

\begin{frame}{Pipelines available in Sequana}
    \begin{textblock*}{5cm}(0.3cm,1.3cm)
        \includegraphics[scale=0.215]{images/pipelines}
    \end{textblock*}
\end{frame}



\begin{frame}{GUI to simplify the usage of snakemake}
    \begin{columns}
        \begin{column}{0.5\textwidth}

            \only<1>{\includegraphics[scale=0.25]{../../images/sequana_init}}

            \only<2>{\includegraphics[scale=0.25]{../../images/choose_pipeline}}

            \only<3>{\includegraphics[scale=0.25]{../../images/choose_input_output}}

            \only<4>{\includegraphics[scale=0.25]{../../images/sequana_pipeline}}

            \only<5>{\includegraphics[scale=0.25]{../../images/sequana_running}}

            \only<6>{\includegraphics[scale=0.25]{../../images/sequana_finish}}

        \end{column}
        \begin{column}{0.5\textwidth}
            \only<1>{
                \begin{itemize}
                    \item Interface developed with PyQT5 and python
                    \item Wrap our snakemake pipelines to ease the usage
                    \item Usable on our cluster, which allows X11
                \end{itemize}
            }
            \only<2-6>{
            \begin{enumerate}
                \item<2-6> Choose a pipeline
                \item<3-6> Set input and output
                \item<4-6> Fill the config formular
                \item<5-6> Run the pipeline
                \item<6> Finished !
            \end{enumerate}
            }
        \end{column}
    \end{columns}
\end{frame}




\section{The Sequana library}

\begin{frame}
\frametitle{Python behind the scene}
 \begin{block}{Python as a glue}
     - Make use of bioconda for requirements and versioning of dependendies. \\
     - For low-level computations, use existing libraries (e.g., pysam)
 \end{block}
 
 \begin{block}{Dev of original tools}
  e.g. coverage or quick taxonomy 
 \end{block}
 
 \begin{block}{add missing bricks}
  - kraken2krona
  - ...
 \end{block}

 \begin{block}{HTML Reporting}
  Based on Jinja and Sequana reports.
 \end{block}
\end{frame}

 
\begin{frame}
\begin{block}{I want to use it}
 module use /pasteur/projets/policy01/Matrix/modules\\
 module load sequana
\end{block} 
\end{frame}


\section{Continuous Integration}

\begin{frame}{Versioning, Test and Documentation}
\begin{tabular}{cp{8cm}}
\vspace{0.5cm}
\includegraphics[width=0.2\textwidth,height=0.1\textheight]{../../images/logo_github.png}
& https://github.com/sequana/sequana\\

\vspace{0.5cm}
\includegraphics[width=0.2\textwidth]{../../images/logo_travis.png}&  
Continuous Integration on Travis with 100 tests with 75\% coverage\\

\vspace{0.5cm}
\includegraphics[width=0.2\textwidth]{../../images/logo_sphinx.png}& 
Uses Sphinx (RST syntax) to document the source 
code and provides user guide.\\

\vspace{0.5cm}
\includegraphics[width=0.2\textwidth]{../../images/logo_rtd.png}& 
Updated after each commits on sequana.readthedocs.io
\end{tabular}

\end{frame}



\section{Summary and Future Directions}

\begin{frame}
 \frametitle{Summary}
Sequana is a versatile tool that provides

\begin{enumerate}
 \item A Python library dedicated to NGS analysis (e.g., tools to visualise standard NGS formats).
 \item A set of pipelines dedicated to NGS in the form of Snakefiles (snakemake)
 \item Original tools to help in the creation of such pipelines including HTML reports.
 \item Standalone applications:
 \begin{itemize}
    \item \textbf{sequana\_coverage} ease the extraction of genomic regions of interest and genome coverage information
 \end{itemize}
\end{enumerate}
\end{frame}


\begin{frame}{Sequana history}
    \includegraphics[scale=0.2]{images/timeline}
    \footnotetext[1]{\tiny Detection and characterization of low and high genome coverage regions using an efficient running median and a double threshold approach.
Dimitri Desvillechabrol, Christiane Bouchier, Sean Kennedy, Thomas Cokelaer
bioRxiv 092478; doi: http://dx.doi.org/10.1101/092478}
\end{frame}

\end{document}
